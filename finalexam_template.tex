\documentclass[12pt]{ctexart}
\usepackage[T1]{fontenc}
\usepackage{CJKutf8}
\usepackage{geometry}
\geometry{verbose,tmargin=2.5cm,bmargin=2.5cm,lmargin=2.5cm,rmargin=2.5cm}
\usepackage{array}
\usepackage{lastpage} % 获取总页数
\usepackage{dashrule} % 用于画虚线
\usepackage{tcolorbox} % 用于画文本框
\usepackage{fancyhdr} % 用于定制页眉页脚

\setlength{\baselineskip}{1.5\baselineskip} % 正文1.5倍行距
\renewcommand{\arraystretch}{1.5} % 表格1.5倍行距

%===页眉页脚定制===%

%---首页页眉页脚---%
\fancypagestyle{plain}{
\fancyhead{}
\fancyfoot[C]{{\footnotesize 第\thepage 页,共\pageref{LastPage} 页}}
\renewcommand{\headrulewidth}{0pt}
}

%---其他页页眉页脚---
\pagestyle{fancy}
\fancyhead[C]{{\footnotesize 中山大学本科生期末考试试卷}}
\fancyfoot[C]{{\footnotesize 第\thepage 页,共\pageref{LastPage} 页}}
\renewcommand{\headrulewidth}{0pt}

\date{}

%===试卷头===%

\begin{document}
\thispagestyle{plain}

\begin{center}
\textbf{\LARGE{}中山大学本科生期末考试}
\par\end{center}{\LARGE \par}

\begin{center}
{\large{}考试科目:逻辑学(A/B卷)}
\par\end{center}{\large \par}

{\kaishu%
\begin{tabular}{>{\raggedright}m{0.5\textwidth}>{\raggedright}m{0.5\textwidth}}
学年学期:2017学年第1学期 & 姓\hspace{2em}名:\underline{\hspace{10em}}\tabularnewline
学\hspace{0.5em}院/系:哲学系 & 学\hspace{2em}号:\underline{\hspace{10em}}\tabularnewline
考试方式:闭卷 & 年级专业:\underline{\hspace{10em}}\tabularnewline
考试时长:120分钟 & 班\hspace{2em}别:\underline{\hspace{10em}}\tabularnewline
\end{tabular}}

\bigskip{}

\noindent \hspace{2em}\tcbox[nobeforeafter,box align=base,left=3pt,right=3pt,top=2pt,bottom=2pt,colback=white!80!black]{\textbf{\large{}警示}}{\fangsong~《中山大学授予学士学位工作细则》第八条:“考试作弊者,不授予学士学位。”
}

\bigskip{}

\hdashrule{5em}{1pt}{2pt}以下为试题区域,共10道题,总分100分,考生请在答题纸上作答\hdashrule{5em}{1pt}{2pt}

%===试题部分===%

\begin{enumerate}
\item (10分)第1题
\item (10分)第2题
\item (10分)第3题
\item (10分)第4题
\item (10分)第5题
\item (10分)第5题
\end{enumerate}

\end{document}
